% Place abstract below.
% Internet provides people all kinds of information and opportunities, however, it also provides chances for hackers to spread their malwares to end users all over the world.

% Software obfuscation is a crucial research field due to the
% severe computer security situation of Internet era. Usually we try to make programs more explicit and clearer for people to read while software obfuscation focuses on 
% complicating source programs to make it difficult or even impossible for attackers to do reverse engineering so that malicious codes and
% software vulnerabilities are constrained from spreading. 

% In this paper we proposed a novel method to do software obfuscation. In source codes, control flow graph was changed with the help of Turing machine encoding.
% Predicates which are determinative for the control flow traversing in source codes are transformed to different Turing machines. These encoded Turing machines will be interpreted by an universal Turing machine when executed. Hence, control flow graph could be complicated to a great extent
% by leveraging this transformation. Software is obfuscated
% through this Turing machine encoding and connecting segmented source programs with Turing Machine models. This paper articulate our Turing machine obfuscation in detail and demonstrate the effectiveness through large volume of experiments.

Obfuscation is an important technique to protect software from adversary
analysis. Control flow obfuscation effectively prevents attackers from
understanding the program structure, hence impeding a broad set of reverse
engineering activities. In this thesis, we propose a novel control flow
obfuscation method which employs Turing machines to simulate the computation of
branch conditions. By weaving the original program with Turing machine
components, program control flow graph and call graph would become more complex.
Moreover, due to the computation complexity of a Turing machine, program
execution flow would become much more complicated and resilient to advanced
reverse engineering approaches through symbolic execution and concolic testing.
%

We have implemented a prototype tool based on the proposed technique. Comparing with
previous work, our control flow obfuscation technique bears three distinct
advantages. 1). Complexity: the complicated implementation of a Turing machine
makes it hard for attackers to understand the program control flow structure.
2). Universality: theoretically, Turing machines can encode any computation. Our
obfuscation is built on top of the LLVM intermediate representation so the
application scope is broadened to almost every language with an LLVM front-end
compiler. 3). Resiliency: our obfuscation is shown to be very resilient to
advanced analysis tools. We have evaluated our method in terms of functionality
correctness, potency, resilience, stealth, and cost, respectively. The
experimental results show that the proposed technique can obfuscate programs in
stealth with good performance and robustness.