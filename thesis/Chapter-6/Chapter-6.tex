\chapter{Discussion}
\label{sec:discussion}
To provide more insights and guideline for further adoption of our proposed
technique, we discuss the multiple aspects of the proposed Turing machine obfuscation
technique in this section.

\section{Complexity}
In general, Turing machine model is a powerful calculator that is capable of
solving any algorithm problem. Note that even a simple operation (e.g., ``add'')
may lead to the change of Turing machine states for hundreds of times; every
``move left'' and ``move right'' operation lead to the tape modification and
``read tape'' or ``write tape'' operations.

Considering Turing machine as a state machine, it is hard---if possible at
all---for adversaries with manual reverse engineering to follow the calculation
logic without understanding the transition table rules and state variables. In
addition, automated deobfuscation tools (e.g., KLEE) can also be defeated due to
the intrinsic complexity of a Turing machine. As reported in our resilience
evaluation (\S\ref{subsec:resilience}), the constraint solver of KLEE failed to
yield proper inputs to cover two of three execution paths.

\section{Application Scope}
Previous obfuscation work~\cite{Sharif} usually targets one or several specific
kinds of predicate expressions. Also, most of them performs source code level
transformations for specific kind of program languages~\cite{Trans}. Turing
obfuscator broadens the application scope to any kind of conditional expression.
In addition, it works for programs written in any language as long as they could
be transformed into the LLVM IR. Considering a large portion of programming
languages have been supported by LLVM, we envision Turning machine obfuscator
would serve to harden many softwares implemented with various kinds of
programming languages.

\section{Branch Selection Techniques}
As previously presented, our current implementation rewrites path condition
instructions to invoke the Turing machine component. While it is mostly
impossible for attackers to reason the semanic of the Turing machine code,
return value of executing the Turing machine component is observable (since the
predicate computation is modeled as a function call to the Turing machine
component). Certain amount of information leakage may become feasible at this
point.

We notice that existing work (\cite{Ma, Maieee}) proposes a different approach
at this step; control flow is directly guided (via \texttt{goto}) to the
selected branch from the neural network obfuscator. While this approach seems to
hide the explicit return value, we argue such technique is not fundamentally
more secure since the hidden return value can be inferred by observing the
execution flow. Another solution that may be employed to protect the predicate
computation result is to use matrix branch logic~\cite{Samjam}. Suppose we model
a branch predicate with a Turing machine function, the general idea is to
further transform Turing machine into a matrix function, and then randomize the
matrix branching function. The involved matrix branch logic and randomness shall
provide additional security consideration at this step. Overall, we argue the
current implementation is reasonable, and we leave it as one future work to
present quantitative analysis of the potential information leakage and
countermeasures at this step.

\section{Execution Overhead}
During the Turing machine computation, frequent state change would indicate lots
of read and write operations. Also, since tape is infinite in Turing machine
model, it needs to allocate enough memory to accommodate complex computations.
In general, the complexity of Turing machine may be considered as a double edge
sword; it impedes adversaries and potentially increases overhead to certain
degree. As reported in the cost evaluation (Fig.~\ref{fig:cost}), we observed
non-negligible performance penalty for both cases. One countermeasure here is to
perform selective obfuscation; users can mark sensitive program codes and guide
the Turing machine obfuscator to only harden those parts. Such strategy would
improve the overall execution speed without sacrificing the major security
considerations.
